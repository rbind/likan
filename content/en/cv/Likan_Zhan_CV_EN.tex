% Options for packages loaded elsewhere
\PassOptionsToPackage{unicode}{hyperref}
\PassOptionsToPackage{hyphens}{url}
\PassOptionsToPackage{dvipsnames,svgnames*,x11names*}{xcolor}
%
\documentclass[
  12pt,
]{article}
\usepackage{lmodern}
\usepackage{amssymb,amsmath}
\usepackage{ifxetex,ifluatex}
\ifnum 0\ifxetex 1\fi\ifluatex 1\fi=0 % if pdftex
  \usepackage[T1]{fontenc}
  \usepackage[utf8]{inputenc}
  \usepackage{textcomp} % provide euro and other symbols
\else % if luatex or xetex
  \usepackage{unicode-math}
  \defaultfontfeatures{Scale=MatchLowercase}
  \defaultfontfeatures[\rmfamily]{Ligatures=TeX,Scale=1}
\fi
\makeatletter
\@ifundefined{KOMAClassName}{% if non-KOMA class
  \IfFileExists{parskip.sty}{%
    \usepackage{parskip}
  }{% else
    \setlength{\parindent}{0pt}
    \setlength{\parskip}{6pt plus 2pt minus 1pt}}
}{% if KOMA class
  \KOMAoptions{parskip=half}}
\makeatother
\usepackage{xcolor}
\IfFileExists{xurl.sty}{\usepackage{xurl}}{} % add URL line breaks if available
\IfFileExists{bookmark.sty}{\usepackage{bookmark}}{\usepackage{hyperref}}
\hypersetup{
  pdftitle={Curriculum Vitae},
  colorlinks=true,
  linkcolor=Maroon,
  filecolor=Maroon,
  citecolor=Blue,
  urlcolor=blue,
  pdfcreator={LaTeX via pandoc}}
\urlstyle{same} % disable monospaced font for URLs
\usepackage[margin=1in]{geometry}
\setlength{\emergencystretch}{3em} % prevent overfull lines
\providecommand{\tightlist}{%
  \setlength{\itemsep}{0pt}\setlength{\parskip}{0pt}}
\setcounter{secnumdepth}{-\maxdimen} % remove section numbering
\usepackage{moresize}  %% more font sizes, ssmall etc
\usepackage[fontsize=13pt]{scrextend}

%%%%%% 1. Change font %%%%%%
\setmainfont{FF Nexus Serif} %% FF Nexus Serif, FF Nexus Sans, Adobe Caslon Pro
\usepackage[fontset=none]{ctex}
\usepackage{xeCJK}
\setCJKmainfont{Source Han Serif SC} %% Kaiti SC; Source Han Serif SC; Source Han Sans SC; simhei.ttf; simfang.ttf

%%%%%% 2. Reduce after enumerate and itemized environment %%%%%%
\let\oldendenumerate=\endenumerate
\def\endenumerate{\oldendenumerate %
   \vspace*{-\baselineskip}%
   }%
   
\let\oldenditemize=\enditemize
\def\enditemize{\oldenditemize %
   \vspace*{-\baselineskip}%
   }%

%%% 3. Used font awesome in latex %%%%
\usepackage[fixed]{fontawesome5} %

%%% Change HyperLink Color %%%%%%%%%%%%%%%%%%%
\usepackage{xcolor} % \definecolor
\definecolor{LinkColor}{HTML}{166E8E} % 0F4C74, 3B699E, 005D88, Elesiver: 166E8E
\hypersetup{urlcolor=LinkColor}

%%%%%%%%%%%%%%%%%%%%%%%%%%%%%%%%%%%%%%%%%%%%%%
%% 2. Change page style
\AtBeginDocument{\let\maketitle\relax} % Supress \maketitle
\usepackage{datetime2}
\usepackage{fancyhdr}
\usepackage{lastpage} % LastPage
\pagestyle{fancy}
\renewcommand{\headrulewidth}{0pt}
\lhead{}
\rhead{} % Likan Zhan's CV
\rfoot{} % Last update: \today
\fancyfoot[C]{--~\thepage/\pageref*{LastPage}~--}

%%%%%%%%%% Remove Warnings: Package microtype Warning: Unknown slot number of character
\makeatletter
\def\MT@warn@unknown{}
\makeatother

% Use upquote if available, for straight quotes in verbatim environments
\IfFileExists{upquote.sty}{\usepackage{upquote}}{}
\IfFileExists{microtype.sty}{% use microtype if available
  \usepackage[]{microtype}
  \UseMicrotypeSet[protrusion]{basicmath} % disable protrusion for tt fonts
}{}


\title{Curriculum Vitae}
\author{}
\date{2016-01-22}

\begin{document}
\maketitle

\hypertarget{likan-zhan}{%
\section{Likan ZHAN}\label{likan-zhan}}

\begin{itemize}
\tightlist
\item
  \faUniversity  ~ No.~15, Xueyuan Road, Haidian District, Beijing
  100083, China
\item
  \faPhone*  ~ + 86 10 8230 3468
\item
  \faEnvelope[regular]  ~
  \href{mailto:zhanlikan@hotmail.com}{\nolinkurl{zhanlikan@hotmail.com}}
\item
  \faGlobe  ~ \url{https://likan.info}
\item
  Last update: 2020-10-21
\end{itemize}

\hypertarget{academic-appointment}{%
\subsection{Academic Appointment}\label{academic-appointment}}

\begin{itemize}
\item
  2018.12 \textasciitilde{} Now, Associate professor,\\
  School of Communication Science,\\
  Beijing Language and Culture University, Beijing, China
\item
  2016.06 \textasciitilde{} 2018.11, Assistant professor,\\
  School of Communication Science,\\
  Beijing Language and Culture University, Beijing, China
\item
  2014.09 \textasciitilde{} 2016.05, Assistant professor,\\
  Faculty of Linguistic Sciences,\\
  Beijing Language and Culture University, Beijing, China
\end{itemize}

\hypertarget{education}{%
\subsection{Education}\label{education}}

\begin{itemize}
\item
  2010.10 \textasciitilde{} 2014.09, Ph.D.~Cognitive Science, \\
  Macquarie University, Sydney, Australia
\item
  2007.09 \textasciitilde{} 2010.07, M.E. Cognitive Psychology, \\
  Beijing Language and Culture University, Beijing, China
\item
  2000.09 \textasciitilde{} 2004.07, B.A. Teaching Chinese as a Second
  Language, \\
  Beijing Language and Culture University, Beijing, China
\end{itemize}

\hypertarget{teaching-experience}{%
\subsection{Teaching Experience}\label{teaching-experience}}

\begin{itemize}
\item
  2019 \textasciitilde{} Now. Experimental Psychology, \\
  Undergraduates, 4 hours per week, 68 hours in total
\item
  2019 \textasciitilde{} Now. Introduction to Neurolinguistics, \\
  Postgraduates, 2 hours per week, 34 hours in total
\item
  2017 \textasciitilde{} Now. Statistics for the Behavioral Sciences, \\
  Undergraduates and Postgraduates, 2 hours per week, 34 hours in total
\item
  2016 \textasciitilde{} Now. R for Modeling and Visualizing Data, \\
  Postgraduates and PhD candidates, 4 hours per week, 68 hours in total
\item
  2015 \textasciitilde{} Now. Introduction to Cognitive Neuroscience, \\
  Postgraduates and PhD candidates, 2 hours per week, 34 hours in total
\item
  2015 \textasciitilde{} Now. Foundations of Scientific Research, \\
  Undergraduates, 3 hours per week, 54 hours in total
\item
  2015 Fall. General Psychology, \\
  Undergraduates, 3 hours per week, 54 hours in total
\end{itemize}

\hypertarget{professional-skills}{%
\subsection{Professional Skills}\label{professional-skills}}

\begin{enumerate}
\def\labelenumi{\arabic{enumi}.}
\tightlist
\item
  Experiment techniques
\end{enumerate}

\begin{itemize}
\item
  Test stimuli: Package \href{https://ctan.org/pkg/pgf?lang=en}{PGF} in
  \href{https://www.latex-project.org}{LaTeX},
  \href{https://imagemagick.org}{ImageMagick} and its implementation in
  R
  \href{https://cran.r-project.org/web/packages/magick/index.html}{magick},
  \href{https://www.omnigroup.com/omnigraffle/}{OmniGraffle}, as well as
  \href{https://www.pixelmator.com}{Pixelmator} for image creating and
  editting; \href{http://www.fon.hum.uva.nl/praat/}{praat}, Packages
  \href{https://cran.r-project.org/web/packages/tuneR/index.html}{tuneR}
  and \href{https://cran.r-project.org/package=seewave}{seewave} in R,
  as well as \href{https://www.adobe.com/products/audition.html}{Adobe
  Audition} for audio editing and processing;
  \href{https://www.blender.org}{blender} and
  \href{https://www.autodesk.com/products/maya/overview}{Autodesk Maya}
  for video and 3d modeling; and \href{https://ffmpeg.org}{ffmpeg} for
  audio and video format converting;
\item
  Experiments building: Be familiar with package
  \href{http://psychtoolbox.org}{Psychtoolbox} under
  \href{https://www.mathworks.com/products/matlab.html}{Matlab},
  \href{https://www.neurobs.com/presentation}{Presentation}, package
  \href{http://www.psychopy.org}{Psychopy} under
  \href{https://www.python.org}{Python},
  \href{https://www.pstnet.com/eprime.cfm}{E-prime}, and
  \href{https://www.sr-research.com/experiment-builder/}{Experiment
  Builder} for script building;
\item
  Eye-tracker: Having advanced experience in using
  \href{https://www.sr-research.com}{Eyelink II/1000 plus} for
  experiment conducting,
  \href{https://www.sr-research.com/experiment-builder/}{Experiment
  Builder} for experiment building, and
  \href{https://www.sr-research.com/data-viewer/}{Data Viewer} for data
  pre-processing;
\item
  E/MEG: In charge of establishing the first Child MEG lab in China; be
  familiar with Yokogawa MEG system for equipment maintenance and MEG
  data acquisition, packages
  \href{http://www.fieldtriptoolbox.org}{fieldtrip} and
  \href{https://github.com/neurodebian/spm12}{SPM12} under Matlab, and
  \href{http://www.besa.de}{BESA} for data processing.
\end{itemize}

\begin{enumerate}
\def\labelenumi{\arabic{enumi}.}
\setcounter{enumi}{1}
\tightlist
\item
  Data processing
\end{enumerate}

\begin{itemize}
\item
  R: Besides base \href{https://www.r-project.org}{R}, be familiar with
  \href{http://r-datatable.com}{data.table},
  \href{https://cran.r-project.org/web/packages/dplyr/index.html}{dplyr},
  and
  \href{https://cran.r-project.org/web/packages/tidyr/index.html}{tidyr}
  etc. for data preprocessing;
  \href{https://stat.ethz.ch/R-manual/R-devel/library/graphics/html/00Index.html}{graphics},
  \href{https://cran.r-project.org/package=lattice}{lattice}, and
  \href{http://ggplot2.tidyverse.org}{ggplot2} etc. for data
  visualization;
  \href{https://cran.r-project.org/web/packages/car/index.html}{car},
  \href{https://github.com/lme4/lme4}{lme4}, and
  \href{https://cran.r-project.org/web/packages/gam/index.html}{gam} for
  computational modeling. Create the R package
  \href{https://github.com/likanzhan/acqr}{acqr} etc. for teaching
  statistics and illustrating data visualization;
\item
  Julia: Recently start to use \href{https://julialang.org}{julia},
  because of the \href{https://julialang.org/benchmarks/}{speed
  limitations} of R. Besides the base and standard libraries, be
  familiar with \href{https://github.com/JuliaData/CSV.jl}{CSV.jl},
  \href{https://github.com/JuliaData/DataFrames.jl}{DataFrames.jl}
  (\href{https://h2oai.github.io/db-benchmark/}{Database benchmark}),
  and
  \href{https://github.com/JuliaData/DataFramesMeta.jl}{DataFramesMeta.jl}
  etc. for data preprocessing;
  \href{https://github.com/JuliaPlots/Plots.jl}{Plots.jl},
  \href{https://github.com/JuliaPlots/StatsPlots.jl}{StatsPlots.jl}, and
  \href{https://github.com/GiovineItalia/Gadfly.jl}{Gadfly.jl} for data
  visualization; as well as
  \href{https://github.com/JuliaStats/Distributions.jl}{Distributions.jl},
  \href{https://github.com/JuliaStats/StatsBase.jl}{StatsBase.jl},
  \href{https://github.com/JuliaStats/StatsModels.jl}{StatsModels.jl},
  \href{https://github.com/JuliaStats/GLM.jl}{GLM.jl}, and
  \href{https://github.com/JuliaStats/MixedModels.jl}{MixedModels.jl}
  for computational modeling;
\end{itemize}

\begin{enumerate}
\def\labelenumi{\arabic{enumi}.}
\setcounter{enumi}{2}
\tightlist
\item
  Other
\end{enumerate}

\begin{itemize}
\item
  Markup languages: Be familiar with
  \href{https://www.latex-project.org}{LaTeX},
  \href{https://daringfireball.net/projects/markdown/}{markdown},
  \href{https://rmarkdown.rstudio.com}{R markdown} (R flavoured
  markdown), \href{https://github.com/fonsp/Pluto.jl}{Pluto.jl} (julia
  package), and \href{https://jupyter.org}{Jupyter} for document
  preparing;
  \href{https://cran.r-project.org/web/packages/knitr/index.html}{knitr}
  (R package)
  \href{https://github.com/JuliaDocs/Documenter.jl}{Documenter.jl} and
  \href{https://github.com/JunoLab/Weave.jl}{Weave.jl} (Julia package)
  for dynamically integrating R or Julia source code into a document;
  \href{https://pandoc.org}{Pandoc} for format conversion;
\item
  Be familiar with the free source-code editor
  \href{https://code.visualstudio.com}{visual studio code};
\item
  Be familiar with the distributed version control system
  \href{https://git-scm.com}{git}, package managers
  \href{https://docs.conda.io/en/latest/}{conda},
  \href{https://brew.sh}{homebrew}, and
  \href{https://www.npmjs.com}{npm};
\item
  Personal website: Use \href{https://gohugo.io}{hugo} under
  \href{https://golang.org}{go} language and the R package
  \href{https://github.com/rstudio/blogdown}{blogdown} to build my
  personal website \url{https://likan.info}.
\end{itemize}

\hypertarget{grants-projects-and-awards}{%
\subsection{Grants, Projects, and
Awards}\label{grants-projects-and-awards}}

\begin{enumerate}
\def\labelenumi{\arabic{enumi}.}
\tightlist
\item
  Grants
\end{enumerate}

\begin{itemize}
\item
  \textbf{Zhan, L.}, Zhou, P., Zhang, L., \& Crain, S. (2019 - 2023).
  The acquisition and online processing of irrealis in Mandarin Chinese.
  \emph{The National Social Science Fund of China}. {[}Grant
  No.~19BYY087{]}. (¥200,000). Role: Principle Investigator.
\item
  \textbf{Zhan, L.}, Qu, Y., Xu, J., Xiao, Y., \& Li, N. (2019 - 2020).
  The eye movements pattern and neural oscitation markers of predictive
  processing in language comprehension. \emph{The Fundamental Research
  Funds for the Central Universities}. {[}Grant No.~19YJ080002{]}.
  (¥70,000). Role: Principle Investigator.
\item
  \textbf{Zhan, L.}, Shi, F., Gao, L., \& Zhang, L. (2019 - 2020). The
  online processing and acquisition of lexical tone in Mandarin Chinese.
  \emph{The Fundamental Research Funds for the Central Universities}.
  {[}Grant No.~18YBT15{]}. (¥70,000). Role: Principle Investigator.
\item
  Shi, D., et al.~(2018 - 2023). Studies of Chinese in Generative
  Grammar and Development of Chinese Grammar in the New Era. \emph{The
  Major Project of the National Social Science Found of China}. {[}Grant
  No.~18ZDA291{]}. (¥800,000). Role: Co-Investigator.
\item
  Si, F., et al.~(2018 - 2023). Language acquisition and language
  cognition under the perspective of syntactic cartography. \emph{The
  Fundamental Research Funds for the Central Universities}. {[}Grant
  No.~18ZDJ06{]}. (¥300,000). Role: Co-Investigator.
\item
  \textbf{Zhan, L.} (2015 - 2016). Experimental explorations of possible
  world semantics. \emph{The Fundamental Research Funds for the Central
  Universities}. {[}Grant No.~15YBB29{]}. (¥20,000). Role: Principle
  Investigator.
\item
  \textbf{Zhan, L.} (2015 - 2016). Introduction to psychology and the
  scientific research. \emph{The Funds Supporting the Growth of the New
  Teachers}. {[}Grant No.~FD201530{]}. (¥7,000). Role: Principle
  Investigator.
\item
  \textbf{Zhan, L.}, Shi, F., Gao, L., \& Zhang, L. (2015 - 2016). The
  Processing and acquisition of tone in Mandarin Chinese. \emph{The
  Fundamental Research Funds for the Central Universities}. {[}Grant
  No.~15YJ050003{]}. (¥70,000). Role: Principle Investigator.
\item
  \textbf{Zhan, L.} (2010 - 2014). The interpretation of conditionals in
  natural language. \emph{Cognitive Science Postgraduate Research Grant
  of Macquarie University}. (\$10,570). Role: Principle Investigator.
\item
  \textbf{Zhan, L.} (20103). The hypothetical property of conditionals
  in natural language. \emph{Macquarie University Postgraduate Research
  Fund}. (\$4,684). Role: Principle Investigator.
\end{itemize}

\begin{enumerate}
\def\labelenumi{\arabic{enumi}.}
\setcounter{enumi}{1}
\tightlist
\item
  Projects
\end{enumerate}

\begin{itemize}
\item
  Zhang, L., et al.~(2018 - Now). The psychological and brain
  development of atypical developing adolescences. \emph{The Program of
  WuTong Innovations Platforms}. Role: Co-Investigator.
\item
  Chen, M., et al.~(2019 - Now). The study of Chinese as a second
  language acquisition and second language teaching (The JunCai Group).
  \emph{The Program of Establishing First-Class Disciplines in BLCU}.
  Role: Co-Investigator.
\end{itemize}

\begin{enumerate}
\def\labelenumi{\arabic{enumi}.}
\setcounter{enumi}{2}
\tightlist
\item
  Awards
\end{enumerate}

\begin{itemize}
\item
  2013.11. The Paula Menyuk Travel Award for the 38th Boston University
  Conference on Language Development. (\$300).
\item
  2013.04. The Travel Award for the 26th Annual CUNY Sentence Processing
  Conference. (\$350).
\end{itemize}

\hypertarget{professional-activities}{%
\subsection{Professional Activities}\label{professional-activities}}

\begin{enumerate}
\def\labelenumi{\arabic{enumi}.}
\item
  Frontiers in Psychology, Review Editor in Language Sciences (2020 - )
\item
  Organization of international meetings
\end{enumerate}

\begin{itemize}
\tightlist
\item
  Academic Advisory Committee, The 24th annual conference of the
  International Association of Chinese Linguistics
  (\href{http://iacl24.blcu.edu.cn}{IACL-24}), Beijing Language and
  Culture University, Beijing, China.
\end{itemize}

\begin{enumerate}
\def\labelenumi{\arabic{enumi}.}
\setcounter{enumi}{2}
\tightlist
\item
  Ad-hoc reviews
\end{enumerate}

\begin{itemize}
\tightlist
\item
  Frontiers in Psychology. (2020 - )
\item
  Scientific Reports. (2018 - 2019);
\item
  Journal of visualized experiments. (2018 - 2019);
\item
  Language Teaching and Linguistic Studies. (In Chinese, 2016);
\item
  SAGE Open. (2016).
\end{itemize}

\begin{enumerate}
\def\labelenumi{\arabic{enumi}.}
\setcounter{enumi}{3}
\tightlist
\item
  Clinical licenses
\end{enumerate}

\begin{itemize}
\item
  Licensed Examiner of Wechsler Preschool and Primary Scale of
  Intelligence 4th Edition (WPPSI-VI) Chinese Version.
\item
  Licensed Examiner of Adaptive Behavior Assessment System 2nd Edition
  (ABAS-II) Chinese Version.
\end{itemize}

\hypertarget{publications}{%
\subsection{Publications}\label{publications}}

\begin{enumerate}
\def\labelenumi{\arabic{enumi}.}
\tightlist
\item
  Book
\end{enumerate}

\begin{itemize}
\tightlist
\item
  \textbf{Zhan, L.} (2015). \emph{The Interpretation of Conditionals in
  Natural Language}. Saarbrucken, Germany: Lap Lambert Academic
  Publishing.
\end{itemize}

\begin{enumerate}
\def\labelenumi{\arabic{enumi}.}
\setcounter{enumi}{1}
\tightlist
\item
  Periodicals
\end{enumerate}

\begin{itemize}
\item
  Zhou, P., Shi, J., \& \textbf{Zhan, L.} (in press). Real-time
  comprehension of garden-path constructions by preschoolers: A Mandarin
  perspective. \emph{Applied Psycholinguistics}.
\item
  Zhou, P., Ma, W., \& \textbf{Zhan, L.} (2019). A deficit in using
  prosodic cues to understand communicative intentions by children with
  autism spectrum disorders: An eye-tracking study. \emph{First
  Language}. doi: 10.1177/0142723719885270
  \href{https://publications.likan.info/Periodicals/FirstLang2019.pdf}{
  \faFilePdf[regular] }
\item
  Zhou, P., \textbf{Zhan, L.}, \& Ma, H. (2019). Understanding others'
  minds: Social inference in preschool children with autism spectrum
  disorder. \emph{Journal of Autism and Developmental Disorders,
  49}(11), 4523-4534. doi: 10.1007/s10803-019-04167-x
  \href{https://publications.likan.info/Periodicals/JAutismDevDisord2019.pdf}{
  \faFilePdf[regular] }
\item
  Zhou, P., \textbf{Zhan, L.}, \& Ma, H. (2019). Predictive language
  processing in preschool children with Autism Spectrum Disorder: An
  eye-tracking study. \emph{Journal of Psycholinguistic Research,
  48}(2), 431-452. doi: 10.1007/s10936-018-9612-5
  \href{https://publications.likan.info/Periodicals/JPsycholinguistRes2018.pdf}{
  \faFilePdf[regular] }
\item
  \textbf{Zhan, L.} (2018). Using eye movements recorded in the visual
  world paradigm to explore the online processing of spoken language.
  \emph{Journal of Visualized Experiments, 140}, e58086. doi:
  10.3791/58086
  \href{https://publications.likan.info/Periodicals/jove-protocol-58086.pdf}{
  \faFilePdf[regular] }
\item
  Zhou, P., Ma, W., \textbf{Zhan, L.}, \& Ma, H (2018). Using the visual
  world paradigm to study sentence comprehension in Mandarin-speaking
  children with autism. \emph{Journal of Visualized Experiments, 140},
  e58452. doi: 10.3791/58452
  \href{https://publications.likan.info/Periodicals/jove-protocol-58452.pdf}{
  \faFilePdf[regular] }
\item
  \textbf{Zhan, L.}, Zhou, P., \& Crain, S. (2018). Using the
  visual-world paradigm to explore the meaning of conditionals in
  natural language. \emph{Language, Cognition and Neuroscience, 33}(8),
  1049-1062. doi: 10.1080/23273798.2018.1448935
  \href{https://publications.likan.info/Periodicals/LangCognNeurosci2018.pdf}{
  \faFilePdf[regular] }
\item
  \textbf{Zhan, L.} (2018). Scalar and ignorance inferences are both
  computed immediately upon encountering the sentential connective: The
  online processing of sentences with disjunction using the visual world
  paradigm. \emph{Frontiers in Psychology, 9}. doi:
  10.3389/fpsyg.2018.00061
  \href{https://www.frontiersin.org/articles/10.3389/fpsyg.2018.00061/full}{
  \faFilePdf[regular] }
\item
  Moscati, V., \textbf{Zhan, L.}, \& Zhou, P. (2017). Children's on-line
  processing of epistemic modals. \emph{Journal of Child Language,
  44}(5), 1025-1040. doi: 10.1017/S0305000916000313
  \href{https://publications.likan.info/Periodicals/JChildLang2016.pdf}{
  \faFilePdf[regular] }
\item
  \textbf{Zhan, L.}, Crain, S., \& Zhou, P. (2015). The online
  processing of only if- and even if- conditional statements:
  Implications for mental models. \emph{Journal of Cognitive Psychology,
  26}(7), 367-379. doi: 10.1080/20445911.2015.1016527
  \href{https://publications.likan.info/Periodicals/JCognPsychol2015.pdf}{
  \faFilePdf[regular] }
\item
  Zhou, P., Crain, S., \& \textbf{Zhan, L.} (2014). Grammatical aspect
  and event recognition in children's online sentence comprehension.
  \emph{Cognition, 133}(1), 262-276. doi:
  10.1016/j.cognition.2014.06.018
  \href{http://publications.likan.info/Periodicals/Cognition2014.pdf}{
  \faFilePdf[regular] }
\item
  Zhou, P., Crain, S., \& \textbf{Zhan, L.} (2012). Sometimes children
  are as good as adults: The pragmatic use of prosody in children's
  on-line sentence processing. \emph{Journal of Memory and Language,
  67}(1), 149-164. doi: 10.1016/j.jml.2012.03.005
  \href{https://publications.likan.info/Periodicals/JMemLang2012.pdf}{
  \faFilePdf[regular] }
\item
  Zhou, P., Su, Y., Crain, S., Gao, L., \& \textbf{Zhan, L.} (2012).
  Children's use of phonological information in ambiguity resolution: A
  view from Mandarin Chinese. \emph{Journal of Child Language, 39}(04),
  687-730. doi: 10.1017/S0305000911000249
  \href{https://publications.likan.info/Periodicals/JChildLang2012.pdf}{
  \faFilePdf[regular] }
\end{itemize}

\begin{enumerate}
\def\labelenumi{\arabic{enumi}.}
\setcounter{enumi}{2}
\tightlist
\item
  Book chapters and conference proceedings
\end{enumerate}

\begin{itemize}
\item
  \textbf{Zhan, L.} (2018). Magnetoencephalography (MEG) as a Technique
  for Imaging Brain Function and Dysfunction. In \emph{Top 10
  Contributions on Psychology} (Chapter 4, pp.~1-38). Telangana, India:
  Avid Science
\item
  \textbf{Zhan, L.}, Crain, S., \& Zhou, P. (2013). The anticipatory
  effects of focus operators: A visual-world paradigm eye-tracking study
  of ``only if'' and ``even if'' conditionals. In N. Goto, K. Otaki, A.
  Sato, \& K. Takita (Eds.), \emph{Proceedings of GLOW in Asia IX 2012}.
  Mie University, Japan.
\end{itemize}

\begin{enumerate}
\def\labelenumi{\arabic{enumi}.}
\setcounter{enumi}{3}
\tightlist
\item
  Peer reviewed conference presentations
\end{enumerate}

\begin{itemize}
\item
  \textbf{Zhan, L.}, \& Zhou, P. (2019, June). \emph{Children differ
  from adults in interpreting disjunctions: Evidence from an
  eye-tracking study}. Poster session presented at Psycholinguistics in
  Iceland - Parsing and Prediction, University of Iceland, Reykjavík,
  Iceland. \href{https://publications.likan.info/Talks/PIPP_Poster.pdf}{
  \faFilePdf[regular] }
\item
  \textbf{Zhan, L.} (2017, September). \emph{Scalar implicature and
  ignorance inference are both locally computed: Evidence from the
  online processing of disjunctions using the visual world paradigm}.
  Oral session presented at the Second High-level Forum on Cognitive
  Linguistics, University of International Business and Economics,
  Beijing, China.
  \href{https://publications.likan.info/Talks/ZhanL2017UIBE.pdf}{
  \faFilePdf[regular] }
\item
  \textbf{Zhan, L.}, Crain, S., \& Zhou, P. (2013, November).
  \emph{Going beyond the information that is perceived: The hypothetical
  property of if-conditionals in Mandarin Chinese}. Oral session
  presented at the Second International Conference on Psycholinguistics
  in China, Fuzhou, China.
\item
  Moscati, V., \textbf{Zhan, L.}, \& Zhou, P. (2013, November).
  \emph{Reasoning on possibilities: An eye tracking study on modal
  knowledge}. Poster session presented at the 38th Annual Boston
  University Conference on Language Development, Boston University, MA.
\item
  Zhou, P., Crain, S., \& \textbf{Zhan, L.} (2013, November).
  \emph{Anticipatory eye movements in children's processing of
  grammatical aspect}. Poster session presented at the 38th Annual
  Boston University Conference on Language Development, Boston
  University, MA.
\item
  \textbf{Zhan, L.}, Crain, S., \& Zhou, P. (2013, March). \emph{The
  hypothetical property of ``if''-statements: A visual-world paradigm
  eye-tracking study}. Poster session presented at CUNY2013: The 26th
  annual CUNY Sentence Processing Conference, Columbia, SC.
\item
  \textbf{Zhan, L.}, Crain, S., \& Zhou, P. (2012, July). \emph{The
  interpretation of conditionals}. Oral session presented at the 7th
  International Conference on Thinking (ICT2012), Birkbeck/UCL, London,
  UK.
\item
  Zhou, P., Crain, S., \& \textbf{Zhan, L.} (2012, March).
  \emph{Children's pragmatic use of prosody in sentence processing}.
  Poster session presented at the 35th Generative Linguistics in the Old
  World (GLOW) Workshop: Production and Perception of
  Prosodically-Encoded Information Structure, University of Potsdam,
  Potsdam, Germany.
\item
  \textbf{Zhan, L.}, Crain, S., \& Khlentzos, D. (2011, August).
  \emph{The basic semantics of conditionals in natural language}. Oral
  session presented at the Harvard-Australia Workshop on Language,
  Learning and Logic, Macquarie University, Sydney, Australia.
\item
  Zhou, P., Crain, S., Gao, L., \& \textbf{Zhan, L.} (2010, September).
  \emph{The role of prosody in children's focus identification}. Oral
  session presented at the Generative Approaches to Language Acquisition
  - North America 4 (GALANA-4), Toronto, Canada.
\item
  Zhou, P., Su, Y., Crain, S., Gao, L., \& \textbf{Zhan, L.} (2010,
  August). \emph{Children's use of prosodic information in ambiguity
  resolution}. Oral session presented at the 8th Conference of
  Generative Linguistics in the Old World Asia (GLOW-in-Asia 8),
  Beijing, China.
\end{itemize}

\begin{enumerate}
\def\labelenumi{\arabic{enumi}.}
\setcounter{enumi}{4}
\tightlist
\item
  Invited talks
\end{enumerate}

\begin{itemize}
\item
  \textbf{Zhan, L.} (2020, October). \emph{Language and Language
  Development}. Invited lecture given at the course of \emph{Language +}
  organized by Prof.~Si.
  \href{http://likan.oss-cn-beijing.aliyuncs.com/Talks/Language_and_Language_Development.pdf}{
  \faFilePdf[regular] }
\item
  \textbf{Zhan, L.} (2019, August). \emph{Gardener or Carpenter: Some
  thoughts on nurturing a child}. Invited presentation given at Smiland
  Daycare Center. \href{https://publications.likan.info/talks/parent/}{
  \footnotesize \faExternalLink* }
\item
  \textbf{Zhan, L.} (2018, December). \emph{Sentential reasoning and
  sentential connectives: Conditional, disjunction, negation, and
  modality}. Invited presentation given at the Workshop of Theoretical
  and Experimental Linguistics, Tsinghua University, Beijing, China.
  \href{https://publications.likan.info/Talks/Sentential_Reasoning_Sentential_Connectives.pdf}{
  \faFilePdf[regular] }
\item
  \textbf{Zhan, L.} (2018, November). \emph{Methods of cognitive
  neuroscience: Focus on language}. Invited presentation given at the
  Child Cognition Laboratory, Department of Foreign Languages and
  Literatures, Tsinghua University, Beijing, China.
  \href{https://publications.likan.info/Talks/MethodsCognNeurosciLang2018NOV.pdf}{
  \faFilePdf[regular] }
\item
  \textbf{Zhan, L.} (2018, November). \emph{Visual world paradigm: An
  eye-tracking technique to study the real time processing of spoken
  Language}. Invited presentation given at the Center for Studies of
  Chinese as a Second Language, Beijing Language and Culture University,
  Beijing, China.
  \href{https://publications.likan.info/Talks/Visual_World_Paradigm.pdf}{
  \faFilePdf[regular] }
\item
  \textbf{Zhan, L.} (2018, October). \emph{Experimental Builder: A
  what-you-see-is-what-you-get tool to build experiment scripts}.
  Invited presentation given at the Center for Studies of Chinese as a
  Second Language, Beijing Language and Culture University, Beijing,
  China.
  \href{https://publications.likan.info/Eyelink_Experiment_Builder_Training_Materials/}{
  \faFilePdf[regular] }
\end{itemize}

\end{document}
