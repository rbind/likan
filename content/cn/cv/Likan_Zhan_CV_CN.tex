\documentclass[12pt,]{article}
\usepackage{lmodern}
\usepackage{amssymb,amsmath}
\usepackage{ifxetex,ifluatex}
\usepackage{fixltx2e} % provides \textsubscript
\ifnum 0\ifxetex 1\fi\ifluatex 1\fi=0 % if pdftex
  \usepackage[T1]{fontenc}
  \usepackage[utf8]{inputenc}
\else % if luatex or xelatex
  \ifxetex
    \usepackage{mathspec}
  \else
    \usepackage{fontspec}
  \fi
  \defaultfontfeatures{Ligatures=TeX,Scale=MatchLowercase}
\fi
% use upquote if available, for straight quotes in verbatim environments
\IfFileExists{upquote.sty}{\usepackage{upquote}}{}
% use microtype if available
\IfFileExists{microtype.sty}{%
\usepackage{microtype}
\UseMicrotypeSet[protrusion]{basicmath} % disable protrusion for tt fonts
}{}
\usepackage[margin=1in]{geometry}
\usepackage{hyperref}
\PassOptionsToPackage{usenames,dvipsnames}{color} % color is loaded by hyperref
\hypersetup{unicode=true,
            pdftitle={简历},
            colorlinks=true,
            linkcolor=Maroon,
            citecolor=Blue,
            urlcolor=blue,
            breaklinks=true}
\urlstyle{same}  % don't use monospace font for urls
\usepackage{graphicx,grffile}
\makeatletter
\def\maxwidth{\ifdim\Gin@nat@width>\linewidth\linewidth\else\Gin@nat@width\fi}
\def\maxheight{\ifdim\Gin@nat@height>\textheight\textheight\else\Gin@nat@height\fi}
\makeatother
% Scale images if necessary, so that they will not overflow the page
% margins by default, and it is still possible to overwrite the defaults
% using explicit options in \includegraphics[width, height, ...]{}
\setkeys{Gin}{width=\maxwidth,height=\maxheight,keepaspectratio}
\IfFileExists{parskip.sty}{%
\usepackage{parskip}
}{% else
\setlength{\parindent}{0pt}
\setlength{\parskip}{6pt plus 2pt minus 1pt}
}
\setlength{\emergencystretch}{3em}  % prevent overfull lines
\providecommand{\tightlist}{%
  \setlength{\itemsep}{0pt}\setlength{\parskip}{0pt}}
\setcounter{secnumdepth}{0}
% Redefines (sub)paragraphs to behave more like sections
\ifx\paragraph\undefined\else
\let\oldparagraph\paragraph
\renewcommand{\paragraph}[1]{\oldparagraph{#1}\mbox{}}
\fi
\ifx\subparagraph\undefined\else
\let\oldsubparagraph\subparagraph
\renewcommand{\subparagraph}[1]{\oldsubparagraph{#1}\mbox{}}
\fi

%%% Use protect on footnotes to avoid problems with footnotes in titles
\let\rmarkdownfootnote\footnote%
\def\footnote{\protect\rmarkdownfootnote}

%%% Change title format to be more compact
\usepackage{titling}

% Create subtitle command for use in maketitle
\providecommand{\subtitle}[1]{
  \posttitle{
    \begin{center}\large#1\end{center}
    }
}

\setlength{\droptitle}{-2em}

  \title{简历}
    \pretitle{\vspace{\droptitle}\centering\huge}
  \posttitle{\par}
    \author{}
    \preauthor{}\postauthor{}
      \predate{\centering\large\emph}
  \postdate{\par}
    \date{2016-01-22}

\usepackage[fontsize=13pt]{scrextend}

%%%%%% 1. Change font %%%%%%
\setmainfont{FF Nexus Serif} % English font: Optima, FF Nexus Serif, Palatino, FF Quadraat Sans Pro, FF Quadraat Pro; Adobe Caslon Pro
\usepackage{ctex}
\setCJKmainfont{Source Han Sans SC} % Source Han Serif SC; STKaiti

%%%%%% 2. Reduce after enumerate and itemized environment %%%%%%
\let\oldendenumerate=\endenumerate
\def\endenumerate{\oldendenumerate %
   \vspace*{-\baselineskip}%
   }%
   
\let\oldenditemize=\enditemize
\def\enditemize{\oldenditemize %
   \vspace*{-\baselineskip}%
   }%

%%% 3. Used font awesome in latex %%%%
\usepackage[fixed]{fontawesome5}
%%%%%%%%%%%%%%%%%%%%%%%%%%%%%%%%%%%%%%%%%%%%%%
%% 2. Change page style
\AtBeginDocument{\let\maketitle\relax} % Supress \maketitle
\usepackage{datetime2}
\usepackage{fancyhdr}
\usepackage{lastpage} % LastPage
\pagestyle{fancy}
\renewcommand{\headrulewidth}{0pt}
\lhead{}
\rhead{} % Likan Zhan's CV
\rfoot{} % Last update: \today
\fancyfoot[C]{--~\thepage/\pageref*{LastPage}~--}

\begin{document}
\maketitle

\hypertarget{section}{%
\section{战立侃}\label{section}}

\begin{itemize}
\tightlist
\item
  \faUniversity ~ 北京市海淀区学院路15号,北京语言大学,综合楼1235室
\item
  \faPhone* ~ + 86 10 8230 3468
\item
  \faEnvelope[regular] ~
  \href{mailto:zhanlikan@hotmail.com}{\nolinkurl{zhanlikan@hotmail.com}}
\item
  \faGlobe ~ \url{https://likan.info}
\item
  最近更新:2019-07-25
\end{itemize}

\hypertarget{section-1}{%
\subsection{职业经历}\label{section-1}}

\begin{itemize}
\item
  2019.07 \textasciitilde{} 现在,副教授,北京语言大学语言康复学院
\item
  2014.10 \textasciitilde{}
  2019.06,助理研究员,北京语言大学语言病理与脑科学研究所
\end{itemize}

\hypertarget{section-2}{%
\subsection{教育经历}\label{section-2}}

\begin{itemize}
\item
  2010.10 \textasciitilde{}
  2014.09,澳大利亚麦考瑞大学,认知科学专业,哲学博士
\item
  2007.09 \textasciitilde{}
  2010.07,北京语言大学,认知心理学专业,教育学硕士
\item
  2000.09 \textasciitilde{}
  2004.07,北京语言大学,对外汉语教学专业,文学学士
\end{itemize}

\hypertarget{section-3}{%
\subsection{教学经历}\label{section-3}}

\begin{itemize}
\item
  2019 \textasciitilde{} 现在,实验心理学,\\
  每周4课时,共68课时,言语听觉本科生
\item
  2017 \textasciitilde{} 现在,行为科学统计学,\\
  每周2课时,共34课时,言语听觉本科生和相关专业硕士研究生
\item
  2016 \textasciitilde{} 现在,R语言与数据统计和数据可视化,\\
  每周4课时,共68课时,相关专业硕士和博士研究生
\item
  2015 \textasciitilde{} 现在,认知神经科学导论,\\
  每周2课时,共34课时,言语听觉本科生和相关专业硕博士研究生
\item
  2015 \textasciitilde{} 现在,医学文献检索与阅读,\\
  每周3课时,共54课时,言语听觉本科生
\item
  2015年秋,普通心理学,\\
  每周3课时,共54课时,言语听觉本科生
\end{itemize}

\hypertarget{section-4}{%
\subsection{专业技能}\label{section-4}}

\begin{enumerate}
\def\labelenumi{\arabic{enumi}.}
\tightlist
\item
  实验技能
\end{enumerate}

\begin{itemize}
\tightlist
\item
  实验材料:能用 \href{https://www.latex-project.org}{LaTeX} 包
  \href{https://ctan.org/pkg/pgf?lang=en}{PGF}、
  \href{https://imagemagick.org}{ImageMagick}、R包
  \href{https://cran.r-project.org/web/packages/magick/index.html}{magick}、
  \href{https://www.omnigroup.com/omnigraffle/}{OmniGraffle}
  (非开源)、和 \href{https://www.pixelmator.com}{Pixelmator} (非开源)
  等设计和编辑测试用图片;用
  \href{http://www.fon.hum.uva.nl/praat/}{praat}、R包
  \href{https://cran.r-project.org/web/packages/tuneR/index.html}{tuneR}、\href{https://cran.r-project.org/package=seewave}{seewave}、和
  \href{https://www.adobe.com/products/audition.html}{Adobe Audition}
  (非开源) 等编辑测试用声音材料;用
  \href{https://www.blender.org}{blender} 和
  \href{https://www.autodesk.com/products/maya/overview}{Autodesk Maya}
  (非开源) 等设计和编辑视频及三维材料;用
  \href{https://ffmpeg.org}{ffmpeg} 等转化视频和音频文件的格式;
\item
  实验设计: 能用 \href{https://www.python.org}{Python} 下的
  \href{http://www.psychopy.org}{Psychopy} 包、
  \href{https://www.mathworks.com/products/matlab.html}{Matlab} 下的
  \href{http://psychtoolbox.org}{Psychtoolbox} 包、
  \href{https://www.pstnet.com/eprime.cfm}{E-prime} (非开源)、
  \href{https://www.neurobs.com/presentation}{Presentation} (非开源), 和
  \href{https://www.sr-research.com/experiment-builder/}{Experiment
  Builder} (非开源) 等设计实验程序;
\item
  眼动技术: 熟悉 \href{https://www.sr-research.com}{Eyelink II/1000
  plus} 型眼动仪、 能用
  \href{https://www.sr-research.com/experiment-builder/}{Experiment
  Builder} 进行实验程序的设计和
  \href{https://www.sr-research.com/data-viewer/}{Data Viewer}
  进行数据分析;
\item
  脑电/磁图仪:主持建立了中国第一个儿童脑磁图实验室; 熟悉日本 Yokogawa
  脑磁图仪的设备维护和数据采集;能用 \href{http://www.besa.de}{BESA}、
  \href{http://www.fieldtriptoolbox.org}{fieldtrip}、和
  \href{https://github.com/neurodebian/spm12}{SPM12}
  等对脑电/磁数据进行分析。
\end{itemize}

\begin{enumerate}
\def\labelenumi{\arabic{enumi}.}
\setcounter{enumi}{1}
\tightlist
\item
  数据分析
\end{enumerate}

\begin{itemize}
\tightlist
\item
  R语言:除了 \href{https://www.r-project.org}{R} 基本包外,擅长用
  \href{http://r-datatable.com}{data.table}、
  \href{https://cran.r-project.org/web/packages/dplyr/index.html}{dplyr}、和
  \href{https://cran.r-project.org/web/packages/tidyr/index.html}{tidyr}
  对数据进行预处理; 用
  \href{https://stat.ethz.ch/R-manual/R-devel/library/graphics/html/00Index.html}{graphics}、
  \href{https://cran.r-project.org/package=lattice}{lattice}、和
  \href{http://ggplot2.tidyverse.org}{ggplot2}
  等对数据进行可视化分析;用
  \href{https://cran.r-project.org/web/packages/car/index.html}{car}、\href{https://github.com/lme4/lme4}{lme4}、和
  \href{https://cran.r-project.org/web/packages/gam/index.html}{gam}
  等进行计算模型模拟;自己编写了R包
  \href{https://github.com/likanzhan/acqr}{acqr}
  以辅助统计教学和数据可视化展示;
\item
  Julia语言:由于R语言在计算速度上的不足,近期开始使用
  \href{https://julialang.org}{julia} 语言,尤其是包
  \href{https://github.com/dmbates/MixedModels.jl}{MixedModels}
  开展数据分析。
\end{itemize}

\begin{enumerate}
\def\labelenumi{\arabic{enumi}.}
\setcounter{enumi}{2}
\tightlist
\item
  其他技能
\end{enumerate}

\begin{itemize}
\tightlist
\item
  标记语言:喜欢用
  \href{https://www.latex-project.org}{LaTeX}、\href{https://daringfireball.net/projects/markdown/}{markdown}\\
  和 \href{https://rmarkdown.rstudio.com}{R markdown} (R包)
  等进行学术写作和排版;用
  \href{https://cran.r-project.org/web/packages/knitr/index.html}{knitr}(R包)把R语言代码嵌入文本写作中;用
  \href{https://pandoc.org}{Pandoc} 在文件的不同格式间进行转换;
\item
  个人网站:用 \href{https://golang.org}{go} 语言下的
  \href{https://gohugo.io}{hugo} 和R语言下的
  \href{https://github.com/rstudio/blogdown}{blogdown}
  包建立了的个人网站 \url{https://likan.info}。
\end{itemize}

\hypertarget{section-5}{%
\subsection{科研项目、合作平台和奖励}\label{section-5}}

\begin{enumerate}
\def\labelenumi{\arabic{enumi}.}
\tightlist
\item
  项目
\end{enumerate}

\begin{itemize}
\item
  2019 \textasciitilde{}
  2023,现代汉语非现实句的在线加工和习得研究,\emph{国家社会科学基金一般项目},{[}批准号:19BYY087{]},20万,主持人
\item
  2019 \textasciitilde{}
  2020,语言理解中预测性加工的眼动和神经振荡模式研究,\emph{北京语言大学院级团队项目},{[}批准号:19YJ080002{]},7万,主持人
\item
  2019 \textasciitilde{}
  2023,生成语法的汉语研究与新时代汉语语法理论创新,\emph{国家社科基金重大项目},{[}批准号:18ZDA291{]},80万,参与者
\item
  2019 \textasciitilde{}
  2020,现代汉语声调加工和习得研究,\emph{北京语言大学校级一般项目},{[}批准号:18YBT15{]},7万,主持人
\item
  2018 \textasciitilde{}
  2023,句法制图视角下的语言习得与认知研究,\emph{北京语言大学校级重大项目},{[}批准号:18ZDJ06{]},
  30万,子项目负责人
\item
  2015 \textasciitilde{}
  2016,可能世界语义学的实验研究,\emph{北京语言大学博士科研启动金},{[}批准号:15YBB29{]},2万,主持人
\item
  2015 \textasciitilde{}
  2016,普通心理学,\emph{北京语言大学新任教师支持基金},{[}批准号:FD201530{]},0.7万,主持人
\item
  2015 \textasciitilde{}
  2016,现代汉语声调习得研究,\emph{北京语言大学院级项目},{[}批准号:15YJ050003{]},7万,主持人
\item
  2010 \textasciitilde{}
  2014,自然语言中条件句的理解,\emph{麦考瑞大学认知科学博士生项目},1万澳元,主持人
\item
  2013.10,自然语言中条件句假设特征研究,\emph{麦考瑞大学博士生研究基金},0.46万澳元,
  主持人
\end{itemize}

\begin{enumerate}
\def\labelenumi{\arabic{enumi}.}
\setcounter{enumi}{1}
\tightlist
\item
  合作平台
\end{enumerate}

\begin{itemize}
\item
  2019 \textasciitilde{}
  现在,汉语作为第二语言的习得和教学研究创新团队,\emph{北京语言大学一流学科团队建设计划(隽才计划)},参与者
\item
  2018 \textasciitilde{}
  现在,特殊儿童青少年的心智发展和脑发育特征研究创新平台,\emph{北京语言大学梧桐创新平台},参与者
\end{itemize}

\begin{enumerate}
\def\labelenumi{\arabic{enumi}.}
\setcounter{enumi}{2}
\tightlist
\item
  奖励
\end{enumerate}

\begin{itemize}
\item
  2013.11:第38届波士顿儿童语言发展研讨会,旅行奖励 300 美元
\item
  2013.04:第26届纽约城市大学句子加工会议,旅行奖励 350 美元
\end{itemize}

\hypertarget{section-6}{%
\subsection{学术兼职}\label{section-6}}

\begin{enumerate}
\def\labelenumi{\arabic{enumi}.}
\tightlist
\item
  会议组织
\end{enumerate}

\begin{itemize}
\tightlist
\item
  国际中国语言学会第24届年会,会务组和咨询委员会委员,2016年7月17日-19日,
  中国北京,北京语言大学
\end{itemize}

\begin{enumerate}
\def\labelenumi{\arabic{enumi}.}
\setcounter{enumi}{1}
\tightlist
\item
  匿名评审
\end{enumerate}

\begin{itemize}
\tightlist
\item
  Scientific Reports. (2018 - 2019)
\item
  Journal of visualized experiments. (2018 - 2019)
\item
  Language Teaching and Linguistic Studies. (In Chinese, 2016)
\item
  SAGE Open. (2016)
\end{itemize}

\begin{enumerate}
\def\labelenumi{\arabic{enumi}.}
\setcounter{enumi}{2}
\tightlist
\item
  执业资格
\end{enumerate}

\begin{itemize}
\item
  韦氏婴幼儿智力量表第四版(WPPSI-VI)中文版主试资格
\item
  适应性行为评定量表第二版(ABAS-II)中文版主试资格
\end{itemize}

\hypertarget{section-7}{%
\subsection{出版物}\label{section-7}}

\begin{enumerate}
\def\labelenumi{\arabic{enumi}.}
\tightlist
\item
  专著
\end{enumerate}

\begin{itemize}
\tightlist
\item
  \textbf{Zhan, L.} (2015). \emph{The Interpretation of Conditionals in
  Natural Language}. Saarbrucken, Germany: Lap Lambert Academic
  Publishing.
\end{itemize}

\begin{enumerate}
\def\labelenumi{\arabic{enumi}.}
\setcounter{enumi}{1}
\tightlist
\item
  杂志论文
\end{enumerate}

\begin{itemize}
\item
  Zhou, P., \textbf{Zhan, L.}, \& Ma, H. (2018). Predictive language
  processing in preschool children with Autism Spectrum Disorder: An
  eye-tracking study. \emph{Journal of Psycholinguistic Research}.
  doi:10.1007/s10936-018-9612-5
  \href{https://publications.likan.info/Periodicals/JPsycholinguistRes2018.pdf}{
  \faFilePdf[regular] }
\item
  \textbf{Zhan, L.} (2018). Using eye movements recorded in the visual
  world paradigm to explore the online processing of spoken language.
  \emph{Journal of Visualized Experiments, 140}, e58086. doi:
  10.3791/58086
  \href{https://publications.likan.info/Periodicals/jove-protocol-58086.pdf}{
  \faFilePdf[regular] }
\item
  Zhou, P., Ma, W., \textbf{Zhan, L.}, \& Ma, H (2018). Using the visual
  world paradigm to study sentence comprehension in Mandarin-speaking
  children with autism. \emph{Journal of Visualized Experiments, 140},
  e58452. doi: 10.3791/58452
  \href{https://publications.likan.info/Periodicals/jove-protocol-58452.pdf}{
  \faFilePdf[regular] }
\item
  \textbf{Zhan, L.}, Zhou, P., \& Crain, S. (2018). Using the
  visual-world paradigm to explore the meaning of conditionals in
  natural language. \emph{Language, Cognition and Neuroscience, 33}(8),
  1049-1062. doi: 10.1080/23273798.2018.1448935
  \href{https://publications.likan.info/Periodicals/LangCognNeurosci2018.pdf}{
  \faFilePdf[regular] }
\item
  \textbf{Zhan, L.} (2018). Scalar and ignorance inferences are both
  computed immediately upon encountering the sentential connective: The
  online processing of sentences with disjunction using the visual world
  paradigm. \emph{Frontiers in Psychology, 9}. doi:
  10.3389/fpsyg.2018.00061
  \href{https://www.frontiersin.org/articles/10.3389/fpsyg.2018.00061/full}{
  \faFilePdf[regular] }
\item
  Moscati, V., \textbf{Zhan, L.}, \& Zhou, P. (2017). Children's on-line
  processing of epistemic modals. \emph{Journal of Child Language,
  44}(5), 1025-1040. doi: 10.1017/S0305000916000313
  \href{https://publications.likan.info/Periodicals/JChildLang2016.pdf}{
  \faFilePdf[regular] }
\item
  \textbf{Zhan, L.}, Crain, S., \& Zhou, P. (2015). The online
  processing of only if- and even if- conditional statements:
  Implications for mental models. \emph{Journal of Cognitive Psychology,
  26}(7), 367-379. doi: 10.1080/20445911.2015.1016527
  \href{https://publications.likan.info/Periodicals/JCognPsychol2015.pdf}{
  \faFilePdf[regular] }
\item
  Zhou, P., Crain, S., \& \textbf{Zhan, L.} (2014). Grammatical aspect
  and event recognition in children's online sentence comprehension.
  \emph{Cognition, 133}(1), 262-276. doi:
  10.1016/j.cognition.2014.06.018
  \href{http://publications.likan.info/Periodicals/Cognition2014.pdf}{
  \faFilePdf[regular] }
\item
  Zhou, P., Crain, S., \& \textbf{Zhan, L.} (2012). Sometimes children
  are as good as adults: The pragmatic use of prosody in children's
  on-line sentence processing. \emph{Journal of Memory and Language,
  67}(1), 149-164. doi: 10.1016/j.jml.2012.03.005
  \href{https://publications.likan.info/Periodicals/JMemLang2012.pdf}{
  \faFilePdf[regular] }
\item
  Zhou, P., Su, Y., Crain, S., Gao, L., \& \textbf{Zhan, L.} (2012).
  Children's use of phonological information in ambiguity resolution: a
  view from Mandarin Chinese. \emph{Journal of Child Language, 39}(04),
  687-730. doi: 10.1017/S0305000911000249
  \href{https://publications.likan.info/Periodicals/JChildLang2012.pdf}{
  \faFilePdf[regular] }
\end{itemize}

\begin{enumerate}
\def\labelenumi{\arabic{enumi}.}
\setcounter{enumi}{2}
\tightlist
\item
  会议论文集
\end{enumerate}

\begin{itemize}
\item
  \textbf{Zhan, L.} (2018). Magnetoencephalography (MEG) as a Technique
  for Imaging Brain Function and Dysfunction. In \emph{Top 10
  Contributions on Psychology} (Chapter 4, pp.~1-38). Telangana, India:
  Avid Science
\item
  \textbf{Zhan, L.}, Crain, S., \& Zhou, P. (2013). The anticipatory e
  ects of focus operators: A visual- world paradigm eye-tracking study
  of ``only if'' and ``even if'' conditionals. In N. Goto, K. Otaki, A.
  Sato, \& K. Takita (Eds.), \emph{Proceedings of GLOW in Asia IX 2012}.
  Mie University, Japan.
\end{itemize}

\begin{enumerate}
\def\labelenumi{\arabic{enumi}.}
\setcounter{enumi}{3}
\tightlist
\item
  同行评审的大会报告
\end{enumerate}

\begin{itemize}
\item
  \textbf{Zhan, L.}, \& Zhou, P. (2019, June). \emph{Children differ
  from adults in interpreting disjunctions: Evidence from an
  eye-tracking study }. Poster session presented at Psycholinguistics in
  Iceland - Parsing and Prediction, University of Iceland, Reykjavík,
  Iceland. \href{https://publications.likan.info/Talks/PIPP_Poster.pdf}{
  \faFilePdf[regular] }
\item
  \textbf{Zhan, L.} (2017, September). \emph{Scalar implicature and
  ignorance inference are both locally computed: Evidence from the
  online processing of disjunctions using the visual world paradigm}.
  Paper presented at the Second High-level Forum on Cognitive
  Linguistics, University of International Business and Economics,
  Beijing, China.
  \href{https://publications.likan.info/Talks/ZhanL2017UIBE.pdf}{
  \faFilePdf[regular] }
\item
  \textbf{Zhan, L.}, Crain, S., \& Zhou, P. (2013, November).
  \emph{Going beyond the information that is perceived: The hypothetical
  property of if-conditionals in Mandarin Chinese}. Paper session
  presented at the Second International Conference on Psycholinguistics
  in China, Fuzhou, China.
\item
  Moscati, V., \textbf{Zhan, L.}, \& Zhou, P. (2013, November).
  \emph{Reasoning on possibilities: An eye tracking study on modal
  knowledge}. Poster session presented at the 38th Annual Boston
  University Conference on Language Development, Boston University, MA.
\item
  Zhou, P., Crain, S., \& \textbf{Zhan, L.} (2013, November).
  \emph{Anticipatory eye movements in children's processing of
  grammatical aspect}. Poster session presented at the 38th Annual
  Boston University Conference on Language Development, Boston
  University, MA.
\item
  \textbf{Zhan, L.}, Crain, S., \& Zhou, P. (2013, March). \emph{The
  hypothetical property of ``if''-statements: A visual- world paradigm
  eye-tracking study}. Poster session presented at CUNY2013: The 26th
  annual CUNY Sentence Processing Conference, Columbia, SC.
\item
  \textbf{Zhan, L.}, Crain, S., \& Zhou, P. (2012, July). \emph{The
  interpretation of conditionals}. Paper presented at the 7th
  International Conference on Thinking (ICT2012), Birkbeck/UCL, London,
  UK.
\item
  Zhou, P., Crain, S., \& \textbf{Zhan, L.} (2012, March).
  \emph{Children's pragmatic use of prosody in sentence processing}.
  Poster session presented at the 35th Generative Linguistics in the Old
  World (GLOW) Workshop: Production and Perception of
  Prosodically-Encoded Information Structure, University of Potsdam,
  Potsdam, Germany.
\item
  \textbf{Zhan, L.}, Crain, S., \& Khlentzos, D. (2011, August).
  \emph{The basic semantics of conditionals in natural language}. Paper
  presented at the Harvard-Australia Workshop on Language, Learning and
  Logic, Macquarie University, Sydney, Australia.
\item
  Zhou, P., Crain, S., Gao, L., \& \textbf{Zhan, L.} (2010, September).
  \emph{The role of prosody in children's focus identification}. Paper
  presented at the Generative Approaches to Language Acquisition - North
  America 4 (GALANA-4), Toronto, Canada.
\item
  Zhou, P., Su, Y., Crain, S., Gao, L., \& \textbf{Zhan, L.} (2010,
  August). \emph{Children's use of prosodic information in ambiguity
  resolution}. Paper presented at the 8th Conference of Generative
  Linguistics in the Old World Asia (GLOW-in-Asia 8), Beijing, China.
\end{itemize}

\begin{enumerate}
\def\labelenumi{\arabic{enumi}.}
\setcounter{enumi}{4}
\tightlist
\item
  受邀报告
\end{enumerate}

\begin{itemize}
\item
  \textbf{Zhan, L.} (2018, December). \emph{Sentential Reasoning and
  Sentential Connectives: Conditional, Disjunction, Negation, and
  Modality}. Inivted presentation given at the Workshop of Theoretical
  and Experimental Linguistics, Tsinghua University, Beijing, China.
  \href{https://publications.likan.info/Talks/Sentential_Reasoning_Sentential_Connectives.pdf}{
  \faFilePdf[regular] }
\item
  \textbf{Zhan, L.} (2018, November). \emph{Methods of Cognitive
  Neuroscience: Focus on Language}. Inivted presentation given at the
  Child Cognition Laboratory, Department of Foreign Languages and
  Literatures, Tsinghua University, Beijing, China.
  \href{https://publications.likan.info/Talks/MethodsCognNeurosciLang2018NOV.pdf}{
  \faFilePdf[regular] }
\item
  \textbf{Zhan, L.} (2018, November). \emph{Visual world paradigm: An
  eye-tracking technique to study the real time processing of spoken
  Language}. Inivted presentation given at the Center for Studies of
  Chinese as a Second Language, Beijing Language and Culture University,
  Beijing, China.
  \href{https://publications.likan.info/Talks/Visual_World_Paradigm.pdf}{
  \faFilePdf[regular] }
\item
  \textbf{Zhan, L.} (2018, October). \emph{Experimental Builder: A
  What-You-See-Is-What-You-Get tool to build experiment scripts}.
  Inivted presentation given at the Center for Studies of Chinese as a
  Second Language, Beijing Language and Culture University, Beijing,
  China.
  \href{https://publications.likan.info/Eyelink_Experiment_Builder_Training_Materials/}{
  \faFilePdf[regular] }
\end{itemize}


\end{document}
